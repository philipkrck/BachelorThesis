\chapter{Methodology and Study Design}
- mehtod triangulation to explore problem space by conducting quantitative 

\section{Feature Selection} \label{section::feature_selection}
- What are the necessary features that were selected and why?
Typical mobile application features include the 
visual and interactive representation of information, horizontal and vertical scrolling, 
communication with a remote API, different means of navigation between screens as well as animations and transitions.
- Why did I choose reduce functionality of original app for means of testing hypotheses?

\section{Performance Comparison} \label{section::performance_comparison_design}
On the one hand, the chosen are the underlying causes of more ephemeral metrics such as page load speed. 
On the other hand, they can be easily measured using software tools as explained in Section.
- restate hypothesis
- find measurement units (CPU, GPU, memory, etc.)
- why are the measurement units relevant?
- think of individual click journeys which will be measured that make use of the features mentioned in section above
- try to find a capture-replay
- which measurement technology will be utilized
- Vorgehen zur Messung / Erhebung
- repeat measurements multiple times (accuracy)


- calculate statistical figures between measurements (variance)


\section{User Experience Comparison} \label{section::usability_comparison_design}
- Contrarily to the verification of the performance hypothesis, 
the usability hypothesis will not be quantitatively, but qualitatively evaluated.
- semi-structured expert interview
- interviews will be conducted by author with individuals
- interviewees are employees of apploft
- every single interviewee has dedicated multiple years of work experience in the
 mobile application space as either a UX/UI designer, software engineer or project manager
- interviews will be conducted via video call
- goal is to get input on subjective  user experience
- guiding questions developed to support interview process based on commonly known UX framework
