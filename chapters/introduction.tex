\chapter{Introduction}
Mobile platforms are dominated by two players - Apple and Google with their respective operating systems iOS and Android. 
Cumulatively, they form a duopoly in the smartphone operating systems market with a combined usage share of 
15.2\% for iOS and 84.8\% for Android in 2020 according to \textcite{IDC2021}.
\\To develop a mobile application for both target platforms, the corresponding development environments and technologies 
are utilized for each platform. This leads to a doubling of cost, development time and 
the need for knowledge of two different application development paradigms. 
This has resulted in the creation of cross-platform (CP) frameworks such as Xamarin (\cite{Xamarin2021}), React Native (\cite{Facebook2021}) and Ionic (\cite{Ionic2021}). 

The fundamental principle behind these frameworks is a specific tech stack operating on a single code base leading to increased development speed
while also having the ability to deploy for both operating systems.

However, CP frameworks lack in terms of performance and 
usability when compared to native technologies as shown by \textcite{Mercado2016} and \textcite{Ebone2018}.
This insufficiency in both aspects can be explained by their inherent architectures (see section \ref{section::other_architectures}). 
Generally, these frameworks utilize a software bridge to communicate with the underlying host platform leading to a transmission delay and thus impaired
app performance. In addition, CP frameworks provide abstract interfaces over multiple native interfaces leading to a decreased subset of functionality
and especially UI customizability. 

Flutter (\cite{FlutterDev20}) is a newly developed open-source UI toolkit developed by Google. 
It has a nonconventional approach to CP development in that every app ships with the Flutter rendering engine. 
Thereby, Flutter bypasses host platform communication in terms of UI rendering and deliveres a natively compiled binary 
which can be executed by the underlying device (see section \ref{section::flutter_architecture}).
Furthermore, the framework delivers a \textit{"[...] collection of visual, structural, platform, and interactive widgets"} (\textcite{GoogleWidgets2021}) for UI customization.
It seems that Flutter addresses the very issues that are generally criticized about CP solutions.

Unfortunately, since Flutter was first released in March 2018 (\cite{FlutterReleases2020}), 
there are no peer reviewed articles comparing the performance or usability to native apps.
\footnote{A search for relevant articles has been conducted using Google Scholar, Sci-hub and IEEE Xplore.} 

\section{Motivation}
As a digital agency specialized on native iOS and Android development, apploft GmbH 
(the partnering company of this thesis) is highly interested in Flutter. 
The implications of using this framework could be wide ranging. 
The services portfolio of apploft could be extended to clients with lower budgets while not being tied to a specific operating system.
For example, startups which are unsure about product market-fit and held by tight budget constraints are especially keen on reaching 
the maximum number of potential customers with their app. Flutter could be utilized for a fast iteration of a product deployable to 
both platforms. 
Worth noting are the benefits of serving smaller customers like startups which include higher growth potential for long-term cooperation
and risk diversification in apploft's client portfolio.

Furthermore, infrastructure setup, package development and app updates would only need to be done for one codebase.

\section{Thesis Goal}
Based on the above stated problems with CP frameworks and the potential business opportunities, the goal of this thesis is to evaluate 
whether Flutter's claims on performance and usability hold up in practice.

\section{Methods}
\label{section:methods}
To properly compare Flutter and native, an application will be developed which has
typical mobile app features including the interaction with a remote API, different means of navigation between screens, 
user authentication and authorization. The feature selection process is further explained in section \ref{section::feature_selection}.

Based on these characteristics, \textit{Kickdown} (\cite{Kickdown2021}) - an online car auction app - was chosen. 
The app has already been developed for iOS by apploft.
To verify laid out claims of the Flutter framework, an app with specific matching functionality and UX design is reproduced.8

\subsection{Performance comparison}
To evaluate performance the typical measures of CPU, GPU and memory usage are chosen in this thesis. 
On the one hand, these metrics are the underlying causes of more ephemeral metrics such as page load speed. 
On the other hand, they can be easily measured using software tools (see section \ref{section::performance_comparison_design}).

\subsection{Usability comparison}
Expert interviews with employees of apploft are conducted to compare the user experience of the developed Flutter app with the iOS application.

\section{Scope \& Limitations}
The feature set of the implemented app is representative for most, but not every type of app. Therefore the results cannot be generalized to 
every type of app.\\
The usability study doesn't have a statistical significance due to its qualitative nature.\\
\textit{TODO: elaborate on these points.}

\section{Plan of Attack}
\label{section:planofattack}
The following is a list of subgoals of this thesis including accompanying deadlines.

\subsection{Section Writing}
\begin{itemize}
    \item \sout{write draft of \textit{Introduction} section and discuss - 05.02.21}
    \item write draft of \textit{Flutter} section and discuss - 19.02.21
    \item write draft of \textit{Study Design} section and discuss - 05.03.21
    \item write draft of \textit{Application Design} section and discuss - 19.03.21
    \item write draft of \textit{Performance Comparison} section and discuss - 02.04.21
    \item write draft of \textit{UX Comparison} section and discuss - 02.04.21
    \item write draft of \textit{Summary} section and discuss - 02.04.21
    \item write draft of \textit{Abstract} section and discuss - 02.04.21
\end{itemize}


\subsection{Performance Measurement}
\begin{itemize}
    \item prepare tooling and conceptualize measurement process - 26.02.
    \item execute a first iteration of measurement process - 05.03.
    \item execute a final iteration of measurement process - 26.03.
\end{itemize}

\subsection{UX Measurement}
\begin{itemize}
    \item prepare interview process - 26.02.
    \item conduct a minimum of 3 interviews - 31.03.
    \item evaluate interviews - 01.04.
\end{itemize}

\subsection{Submission}
\begin{itemize}
    \item topic submission - 15.02.
    \item send out thesis for proof reading - 05.04.21
    \item submit to examination office + register for colloquium - 09.04.
\end{itemize}

*The minimum requirement is to complete building out the \textit{offerings} screen. 
This is the most complex screen of the app and constitutes the main feature. 
It is sufficient for performance comparison as well as a usability study. 
However, if time permits, more of the app will be developed and comparatively evaluated. \\

\section{Following Chapter Summary}
\textit{- TODO: Describe structure of thesis and summarize each chapter. 
This should be somehow interwoven into introduction.}