\chapter{Introduction}
\label{section:introduction}
Mobile platforms are dominated by Apple and Google. Their respective operating systems, iOS and Android form a duopoly in the smartphone operating systems market with a combined usage share of 
15.2\% for iOS and 84.8\% for Android in 2020 (\cite{IDC2021}).
\\To develop a mobile software application (\textit{app}) for both target platforms, the corresponding development environments and technologies 
are utilized for each platform. This leads to a doubling of cost, development time, and 
the need for knowledge of two different application development paradigms. 
As a result, cross-platform frameworks such as Xamarin (\cite{Xamarin2021}), React Native (\cite{Facebook2021}), and Ionic (\cite{Ionic2021}) have been created. 

%% MOTIVATION FOR CROSS-PLATFORM FRAMEWORKS GENERALLY
The fundamental principle behind these frameworks is the provision of a unified tech stack operating on a single code base leading to increased development speed
while also providing the ability to deploy for both mobile operating systems.

%% PROBLEM WITH CROSS-PLATFORM FRAMEWORKS
Generally, cross-platform frameworks utilize \textit{web views}\footnote{Web views are UI component in both iOS and Android to display web content such as \textit{HTML}, \textit{CSS} and \textit{JavaScript}} or a software bridge to communicate with the underlying host platform plugging into 
native interfaces. In both cases, communication channel delays may occur during app runtime leading to reduced execution speed. In addition, 
cross-platform frameworks deliver abstract interfaces over multiple native interfaces leading to a decreased subset of functionality
and especially impaired UI customizability. These inherent architecture attributes (further examined in Section \ref{section::other_architectures}) explain
both impeded performance (\cite{Ebone2018} and \cite{Corbalan2019}) and user experience (\cite{Mercado2016} and \cite{Angulo2014}) compared to native technologies.

%% INTRODUCE FLUTTER
However, over the past 2 years one particular cross-platform technology with certain distinct attributes has risen strongly in popularity (\cite{Statista2021}):
Flutter (\cite{FlutterDev20}) is a newly developed open-source UI toolkit by Google. 
It has a nonconventional approach to cross-platform development in that every app ships with the framework's rendering engine (Section \ref{section::flutter_architecture}). 
Thereby, Flutter bypasses host platform communication by avoiding the underlying system app SDK for UI rendering.
Flutter apps are compiled in native binary format which can be directly executed by the mobile computing device (see Section \ref{section::flutter_architecture}).
Furthermore, the framework provides a \textit{"[...] collection of visual, structural, platform, and interactive widgets"} (\cite[l.1]{GoogleWidgets2021}) for UI customization.
It seems that Flutter addresses the exact same issues that are generally criticized about cross-platform solutions.

\section{Motivation}
\label{section:motivation}
As a digital agency specialized on native iOS and Android development, \textbf{apploft GmbH} 
(the partnering company of this thesis; \cite{apploft2021}) is highly interested in Flutter. 
The implications of using this framework could be wide-ranging, including the extension of the services portfolio
to clients with lower budgets.
For example, startups which are unsure about product/market fit (\cite{Andreesen2007}) and held by tight budget constraints are especially keen on reaching 
the maximum number of potential customers with their app. Flutter could be utilized for a fast iteration of a product deployable to 
both mobile platforms. 
Worth noting are the benefits of serving smaller customers like startups which include higher growth potential for long-term cooperation
and risk diversification in apploft's client portfolio.\\
In addion, instead of focusing on platform customization with native tooling, more effort could 
be directed into developing unique custom features including animations based on advanced motion design.
Furthermore, infrastructure setup, package development and app updates would only be necessary for one codebase.\\
Since the aforementioned economic incentives are not necessarily company-specific, they are relevant for mobile application developers at large.
As such, this thesis may contribute to future work on Flutter's architecture attributes and their implications on runtime performance and UI rendering of frontend toolkits.

\section{Problem Statement}
To the best of the author's knowledge, there are no peer reviewed articles comparing the performance or usability to native apps\footnote{A search for relevant articles has been conducted using Google Scholar, Sci-hub and IEEE Xplore.}.
This leaves a gap in the current literature as both aspects are known challenges of cross-platform frameworks (\cite{BioernHansen2019}) that must be addressed for Flutter specifically to determine in what capacity the framework may be used for building mobile applications.

\section{Thesis Objective} \label{section::thesis_objective}
This thesis will focus on comparing the (1) performance and (2) usability of Flutter's framework technology to iOS specifically. However, to make the comparison, it is assumed that mimicking Android-specific appearance and behavior is rather straightforward as 
both Flutter and Android utilize Google's \textbf{Material design} (\cite{Google2021}) for their default components.
Additionally, testing the framework against iOS is of particular interest as Flutter is less likely
to exploit system specific properties for performance optimization in Apple's closed operating system.\\
The aims of this thesis are to address the initially stated problems with cross-platform frameworks and to independently validate Flutter's marketing claims.
Specifically, Google's assertions that Flutter can match \textit{"native performance"} and the framework can be utilized for building \textit{"expressive and flexible UI"} (\cite{FlutterDev20}),
which will serve as inductively derived hypotheses for empirical verification:

\textbf{$H_P$}: The Flutter framework yields comparable \textbf{performance} to native iOS  app development frameworks for iPhone.

\textbf{$H_U$}: Comparable \textbf{user experiences} can be created with Flutter and native iOS frameworks for iPhone.

\section{Methods} \label{section::methods}
An archetypal native mobile app has been chosen as a case study for the evaluation of both hypotheses $H_P$ and $H_U$.
Based on typical mobile application facets (explained in detail in Section \ref{section::facet_selection}), \textit{Kickdown} (\cite{Kickdown2021}) - an online car auction app - was chosen for this thesis. 
The app has already been developed for iOS by apploft and released to the App Store in February of 2021 (\cite{Apple2021e}).
For the purposes of this thesis, a Flutter clone has been developed to mimick the relevant subset of UI and functionality found in the original app (see Section \ref{subsection::clone_app_feature_reduction}). 
Both the original and Flutter clone app are used for subsequent hypothesis testing.


\subsection{Performance comparison}
The assessment of the performance hypothesis ($H_P$) has been conducted by profiling specific performance metrics including CPU, GPU, and memory usage 
for particular use case flows in the original and Flutter application. 
The profiling results have been analyzed using common statistical techniques.
A further explanation of the measurement process is detailed in Section~\ref{section::performance_comparison_design}.

\subsection{Usability comparison}
The analysis of the more qualitative usability hypothesis ($H_U$) has been evaluated by the means of semi-structured expert interviews.
Specifically, study participants have been asked to evaluate both the Kickdown iOS and the Flutter application clone along various metrics.

A detailed explanation of the interview process is provided in Section \ref{section::usability_comparison_design}.

\section{Scope \& Limitations}
The feature set of the implemented app is considered to be representative for most, but not all types of apps (see Section \ref{section::facet_selection}). 
Therefore, although the results cannot be generalized to all types of apps, they should be seen as representative indicators of Flutters value as a cross-platform framework. 
Furthermore, the deductively chosen methodology yields the potential of finding adjacent hypothesis (mentioned in Chapter \ref{chapter::summary}) which may be
explored in future research.\\
Despite the qualitative nature of the usability study, the depth of detail gained from the expert interviews
should not be neglected in favor of exclusively quantitative methods. Many previously overlooked considerations were suggested by the interviewees.\\
Given the current research state on the Flutter framework, the insights provided by the subject matter  experts interviewed is especially compelling.


\section{Thesis Structure}
The ensuing chapters are structured as follows:
In Chapter \ref{chapter::flutter}, the reader is given the necessary background knowledge on the Flutter framework to contextualize and understand this thesis. Chapter \ref{chapter::study_design} presents the methodology to evaluate the research goal. 
Succeeding the methodological elaboration follows a discourse on the application design and implementation of the Flutter clone app (Chapter \ref{chapter::implementation})
Chapter \ref{chapter::results} outlines and discusses the results of the study.
Finally, key findings are summarized, related back to the research question and future research opportunities are proposed (Chapter \ref{chapter::summary}).
