\chapter{Introduction}
Mobile platforms are dominated by two players - Apple and Google with their respective operating systems iOS and Android. 
Cumulatively, they form a duopoly in the smartphone operating systems market with a combined usage shares of 
15.2\% for iOS and 84.8\% for Android in 2020 according to \textcite{IDC2021}.
\\To develop a mobile application for both target platforms, the corresponding development environments and technologies 
are utilized for each platform. This leads to a doubling of cost, development time and 
the need for knowledge of two different application development paradigms. 
This has resulted in the creation of cross platform frameworks such as Xamarin (\cite{Xamarin2021}), React Native (\cite{Facebook2021}) and Ionic (\cite{Ionic2021}). 

The premise of these frameworks is a tech stack operating on a single code base leading to increased development speed
while also having the ability to deploy for both operating systems.

However, platform-specific development, these cross-platform frameworks lack in terms of performance and 
usability when compared to native technologies as shown by \textcite{Mercado2016} and \textcite{Ebone2018}.

Flutter claims to solve both of these issues. It is an open-source cross-platform UI toolkit developed by Google for building 
\textit{"[...] natively compiled applications for mobile, web and desktop from a single code base"} (\cite{FlutterDev20}). 
The main value proposition of Flutter is native performance by compiling to platform specific code 
while also providing the ability to develop expressive and flexible UI designs.

If these claims hold true, there could be shift in terms of usage of Flutter by app developers. 
Unfortunately, since Flutter was first released in March 2018 (\cite{FlutterReleases2020}), 
there are no peer reviewed articles comparing the performance or usability to native apps.
\footnote{An extensive search for relevant articles has been conducted using Google Scholar, Sci-hub and IEEE Xplore.} 

\section{Motivation}
As a digital agency specialized on native iOS and Android development, apploft GmbH is highly interested in Flutter. 
The implications of using this framework could be wide ranging. 
The services portfolio of apploft could be extended to clients with lower budgets while not being tied to a specific operating system. 

Furthermore, infrastructure setup, package development and app updates would only need to be done for one codebase.

\section{Thesis Goal}
Based on the above stated problem and the potential business implications, the goal of this thesis is to evaluate 
whether Flutters claims on performance, and usability hold up in practice.

\section{Methods}
\label{section:methods}
To properly compare Flutter and native, an application will be developed which has typical mobile app features 
including the interaction with a remote API, user authentication and authorization and different means of navigation between screens.

Based on these characteristics, \textit{Kickdown} - an online car auction app was chosen. 
The app is already developed for iOS by apploft.
To verify laid out claims of the Flutter framework an exact clone is built to compare performance and usability characteristics.

\subsection{Performance comparison}
To evaluate performance the typical measures of CPU, GPU and memory usage are chosen in this paper. 
On the one hand these metrics are the underlying causes of more ephemeral metrics such as page load speed 
apart from software implementation complexity. On the other hand, these metrics can be easily measured using software tools.

\subsection{Usability comparison}
Expert interviews with employees of apploft are conducted to compare the user experience of the developed Flutter app with the iOS application.

\section{Scope \& Limitations}
The feature set of the implemented app is representative for most, but not every type of app. Therefore the results cannot be generalized to 
every type of app.
The usability study doesn't have a statistical significance due to its qualitative nature.



The feature variance of mobile applications is rather high. 
Features beyond those mentioned in ~\ref{section:methods} include on-device machine learning, augmented reality and more. 
These types of features will be intentionally excluded from the app, 
due to the high implementation effort which would exceed the scope of this thesis.\\
\textit{If a usability study is conducted, N may be too small to have a statistical significance.}

\section{Plan of Attack}
\label{section:planofattack}
The following is a list of subgoals of this thesis including accompanying deadlines.

\subsection{Section Writing}
\begin{itemize}
    \item \sout{write draft of \textit{Introduction} section and discuss - 05.02.21}
    \item write draft of \textit{Flutter} section and discuss - 19.02.21
    \item write draft of \textit{Study Design} section and discuss - 05.03.21
    \item write draft of \textit{Application Design} section and discuss - 19.03.21
    \item write draft of \textit{Performance Comparison} section and discuss - 02.04.21
    \item write draft of \textit{UX Comparison} section and discuss - 02.04.21
    \item write draft of \textit{Summary} section and discuss - 02.04.21
    \item write draft of \textit{Abstract} section and discuss - 02.04.21
\end{itemize}


\subsection{Performance Measurement}
\begin{itemize}
    \item prepare tooling and conceptualize measurement process - 26.02.
    \item execute a first iteration of measurement process - 05.03.
    \item execute a final iteration of measurement process - 26.03.
\end{itemize}

\subsection{UX Measurement}
\begin{itemize}
    \item prepare interview process - 26.02.
    \item conduct min 3 interviews - 31.03.
    \item evaluate interviews - 01.04.
\end{itemize}

\subsection{Submission}
\begin{itemize}
    \item topic submission - 15.02.
    \item send out thesis for proof reading - 05.04.21
    \item submit to examination office + register for colloquium - 09.04.
\end{itemize}

*It is unclear how fast the author will be able to implement the features of the app. 
The minimum requirement is to complete building out the \textit{offerings} screen. 
This is the most complex screen of the app and constitutes the main feature. 
It is sufficient for performance comparison as well as a usability study. 
However, if time permits, more of the app will be developed and comparatively evaluated. \\

\subsection{Following Chapter Summary}
\textit{- TODO: Describe structure of thesis and summarize each chapter. 
This should be somehow interwoven into introduction.}