\chapter{Application Design and Implementation} \label{chapter::implementation}
This thesis does not use the engineering method (see \cite{Ertas1996}) for evaluating the hypotheses (Section \ref{section::thesis_objective}) and thus application design and implementation will be explained from a perspective of
a necessary requirement to execute both the performance and UX comparison (detailed in Chapter\ref{chapter::study_design}).
Naturally, software implementation complexity may impede performance if unscalable algorithms or memory-inefficient datastructures are chosen.
Therefore, the goal from a development perspective was to implement the Flutter app as closely as possible to the original app in order to facilitate a fair comparison between both apps.\\
Firstly, general implementation differences resulting from Flutter's declarative nature are explained and contrasted to UIKit's imperative approach.
Secondly, the usage of the Cupertino package and its UI components in the clone application is inspected. 
Finally, the Image Gallery architecture is examined as a particularly difficult section of the app in both implementations.\\


\section{Technicalities}
- what is the exact Flutter version that I've been using?\\
- what is the exact Dart version that I've been using?

\section{Declarative vs. Imperative UI}


\section{Flutter UI Component Usage}
- used as many out of the box cupertino widgets as possible for implementation\\
- only deviated from above guideline when absolutely necessary\\
    - this was the case for the more view UI which had to be custom built from many smaller Flutter widgets.\\
    - this was also the case for the image gallery which is explained in next section


\section{Image Gallery Architecture}
- image gallery (zooming ability), image caching

