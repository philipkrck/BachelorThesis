\chapter{Application Design and Implementation} \label{chapter::implementation}
Thesis is not an [Ingenieurswissenschaftliche Arbeit] and thus application design and implementation will be explained from a perspective
a necessary requirement to execute both the performance and UX comparison.
Generally, goal from development perspective was to get implement Flutter as close as possible to the original app to facilitate a fair comparison between both apps (as explained in \ref{}).
Firstly, general implementation differences resulting from Flutter's declarative nature will be explained and contrasted to UIKit's imperative approach.
Secondly, the usage of the Cupertino package and its UI components in the clone application is inspected. 
Finally, the Image Gallery architecture is examined as a particularly difficult section of the app in both implementations.\\


\section{Technicalities}
- what is the exact Flutter version that I've been using?
- what is the exact Dart version that I've been using?




\section{Flutter Implementation}

- 
- 
- used as many out of the box cupertino widgets as possible for implementation
- only deviated from above guideline when absolutely necessary
- this was the case when implementing the following features
- image gallery (zooming ability), image caching, custom modal iOS 13 presentation)s
- Naturally, software implementation complexity may impede performance if unscalable algorithms or memory-inefficient datastructures are chosen.
In order to prevent this bias to be introduced into the experiment, the Flutter app was implemented as closely as possible to the original iOS app. Todo: explain in what way.

- should I explain the general app architecture which was used?(MVVM + Services)

